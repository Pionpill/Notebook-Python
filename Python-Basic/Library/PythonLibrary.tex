\documentclass{PionpillNote-book}

\import{Parts/style}{tikz-style.tex}

\title{Python Library}
\author{
    Pionpill \footnote{笔名:北岸,电子邮件:673486387@qq.com,Github:\url{https://github.com/Pionpill}} \\
    本文为 Python 相关的内置方法,模块做的简单笔记\\
}

\date{\today}

\begin{document}

\pagestyle{plain}
\maketitle

\noindent\textbf{前言:}

笔者为软件工程系在校本科生,主要利用 Python 进行数据科学与机器学习使用,也有一定游戏开发经验。

本文主要记录 Python 常用的一些内置函数,魔法函数,标准库,装饰器等用法。

在阅读本篇笔记时,本人默认读者已经看过我的 <<Fluent Python>> 学习笔记或对 Python 有一定程度的了解。本文默认读者都具备中阶 Python 语法。这篇文章写完也就可以进入高阶了捏。

本人的书写环境:
\begin{itemize}
    \item Python: 3.8.5
\end{itemize}

\date{\today}
\tableofcontents
\newpage

\setcounter{page}{1} 
\pagestyle{fancy}

\part{内置函数}
\import{Parts/Part-1/Chapter-Magic}{Math.tex}
\import{Parts/Part-1/Chapter-Magic}{Class.tex}
\import{Parts/Part-1/Chapter-Magic}{Other.tex}

\part{Python 标准库}
\import{Parts/Part-2/Chapter-File}{Json.tex}



\end{document}

