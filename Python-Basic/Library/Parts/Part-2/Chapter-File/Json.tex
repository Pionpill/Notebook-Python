\chapter{文件库}
\section{Json}

Json 模块是用于读写 json 文件的一个轻量的标准模块。导入方式及内容如下:

\begin{python}
>>> import json
>>> dir(json)
['JSONDecodeError', 'JSONDecoder', 'JSONEncoder', '__all__', '__author__', '__builtins__', '__cached__', '__doc__', '__file__', '__loader__', '__name__', '__package__', '__path__', '__spec__', '__version__', '_default_decoder', '_default_encoder', 'codecs', 'decoder', 'detect_encoding', 'dump', 'dumps', 'encoder', 'load', 'loads', 'scanner']
>>> json.__all__
['dump', 'dumps', 'load', 'loads', 'JSONDecoder', 'JSONDecodeError', 'JSONEncoder']
\end{python}

\subsection{基础用法}

Json 标准库主要提供了四个方法: \texttt{dumps, dump, loads, load} 其中,\texttt{dumps, loads} 函数不涉及文件,\texttt{dump, load} 涉及文件。

\subsubsection{\texttt{dumps, dumps}}

\texttt{dump} 与 \texttt{dumps} 函数用于对 Python 对象进行序列化。将一个 Python 对象序列化为 JSON 格式的编码。

\texttt{dump} 函数定义:
\begin{python}
dump(obj, fp, *, skipkeys=False, ensure_ascii=True, check_circular=True, allow_nan=True, cls=None, indent=None, separators=None, default=None, sort_keys=False, **kw)
\end{python}

其中各个参数的意义如下:
\begin{itemize}
    \item \textbf{obj}: 要序列化的对象
    \item \textbf{fp}: 文件描述符,将序列化的 str 保存到文件中。
    \item skipkeys: 默认为 False,若为 True,则跳过非基本类型的 dict 键。
    \item ensure\_ascii: 默认为 True,将所有传入的非 ASCII 字符转义输出,False 则原样输出。
    \item check\_circular: 默认为 True,若为 False 则跳过对容器类型的循环引用检查。
    \item allow\_nan: 默认为 True,如果为 False 则严格遵守 JSON 规范,引发一些错误,若为 True,则使用错误对象的  JavaScript 等效值。
    \item indent: 缩进格式,默认最紧凑的方式缩进。
    \item separators: 去除分隔符后面的空格。
    \item default: 如果无法序列化,调用对应函数处理。
    \item sort\_key: 如果为 True,则输出按键值排序。
\end{itemize}

\texttt{dumps} 函数定义 
\begin{python}
dumps(obj, *, skipkeys=False, ensure_ascii=True, check_circular=True, allow_nan=True, cls=None, indent=None, separators=None, default=None, sort_keys=False, **kw)
\end{python}

\texttt{dumps} 函数除了没有参数 \texttt{fp} 其他和 \texttt{dump} 函数相同。

\subsubsection{\texttt{load, loads}}

\texttt{load, loads} 函数使用反序列的方法将 json 对象解码为 python 对象。

\texttt{load} 函数如下:

\begin{python}
load(fp, *, cls=None, object_hook=None, parse_float=None, parse_int=None, parse_constant=None, object_pairs_hook=None, **kw)
\end{python}

其中各个参数意义如下:

\begin{itemize}
    \item \textbf{fp}: 文件描述符。
    \item object\_hook: 可选函数,用于实现自定义解码器。指定一个函数,该函数负责把反序列化后的基本类型对象转换成自定义类型的对象。
    \item parse\_float: 用于对 float 字符串进行解码。
    \item parse\_int: 用于对 int 字符串进行解码。
    \item parse\_constant: 用于对 -Infinity Infinity NaN 字符串进行调用。
    \item object\_pairs\_hook: 可选函数,暂时不知道干什么的。
\end{itemize}

\texttt{loads} 函数如下:

\begin{python}
loads(s, *, cls=None, object_hook=None, parse_float=None, parse_int=None, parse_constant=None, object_pairs_hook=None, **kw)
\end{python}

其中部分参数意义如下:
\begin{itemize}
    \item \textbf{s}: 将 s(包含 JSON 文档的 str,bytes,bytearray 实例) 反序列化为 Python 对象。
\end{itemize}

\subsection{转换规则}

Python 原始类型向 Json 类型的转换对照表如下:
\begin{table}[H]
    \centering
    \caption{json 转换表}
    \label{table:json 转换表}
    \setlength{\tabcolsep}{10mm}
    \begin{tabular}{c|c}
        \toprule
        \textbf{Python} & \textbf{Json} \\
        \midrule
        dict & object \\
        list,tuple & array \\
        str,unicode & string \\
        int,long,float & number \\
        True/False & true/false \\
        None & null \\
        \bottomrule
    \end{tabular}
\end{table}