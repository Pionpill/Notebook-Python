\section{文本和字节序列}
\subsection{字符问题}

``字符串''是个简单的概念:一串字符组成的序列,从 Python3 的 \texttt{str} 对象中获取的元素时 Unicode 字符,这相当于从 Python2 的 \texttt{unicode} 对象中获取的元素,而不是从 Python2 的 \texttt{str} 对象中获取的原始字符序列。

Unicode 标准把字符的标识和具体的字节表述进行了如下明确区分。
\begin{itemize}
    \item 字符的标识,即码位,时 0-1114111的数字(十进制),在 Unicode 标准中以 4-6 个十六进制数字表示,前缀 ``U+''。
    \item 字符的具体表述取决于所用的编码,编码时在码位和字节序列之间转换时使用的算法。
\end{itemize}

把码位转换成字节序列的过程时编码,把字节序列转换成码位的过程时解码。

\lstinputlisting[language=Python]{../../Scripts/DataStruct/4-1.py}

\subsection{字节概要}

Python 内置了两种基本的二进制序列类型:Python3 引入的不可变的 \texttt{bytes} 类型和 Python2.6 添加的可变 \texttt{bytearray} 类型。\texttt{bytes} 和 \texttt{bytearray} 对象的各个元素时介于 0-255 之间的整数。  

\lstinputlisting[language=Python]{../../Scripts/DataStruct/4-2.py}

和许多序列类型一样,\texttt{seq[0]} 返回的是元素的类型,而 \texttt{seq[index:index+1]} 返回的则是 \texttt{seq} 类型,往往 \texttt{seq[0] == seq[:1]} 是成立的,但也仅对于 seq 这个类型成立。

二进制序列各个字节的值可能会使用以下三种不同的方式显示:
\begin{itemize}
    \item 可打印的 ASCII 范围内的文字,使用 ASCII 本身,也即数字
    \item 制表符,换行符等,使用转义序列\textbackslash t,\textbackslash n。
    \item 其他字节类型,使用十六进制转义序列,例如 (\textbackslash x00 是空字节)。
\end{itemize}

除了格式化方法和几个特殊处理 Unicode 数据的方法之外,\texttt{str} 类型的其他方法都支持二进制序列。\texttt{re} 模块的正则表达式也支持对二进制序列的处理。

二进制序列有个方法是 \texttt{str} 没有的,名为 \texttt{fromhex},作用是解析十六进制数字对(数字对中间的空格可选),构建新的二进制序列。

\begin{python}
bytes.fromhex('31 4B CE A9')    # b'1K\xce\xa9'
\end{python}

构建 bytes 或 bytearray 示例还可以调用各自的构造方法,传入下述参数:
\begin{itemize}
    \item 一个 str 对象和一个 encoding 关键字参数。
    \item 一个可迭代对象,提供 0-255 之间的数值。
    \item 一个整数的,使用空字节创建对应长度的二进制序列(已过失)。
    \item 一个实现了缓冲协议的对象,将源对象中的字节序列复制到新建的二进制序列中。
\end{itemize}

使用缓冲类对象构建二进制序列是一种低层操作,可能涉及类型转换。

\lstinputlisting[language=Python]{../../Scripts/DataStruct/4-3.py}

\noindent\textbf{结构体和内存视图}

\texttt{struct} 模块提供了一些函数,把打包的字节序列转换成不同类型字段组成的元组,还有一些函数用于执行反向转换,把元组转换成打包的字节序列。\texttt{struct}模块能处理 \texttt{bytes},\texttt{bytearray},\texttt{memoryview}对象。

\subsection{基本的编解码器}

Python 自带了超 100 中编解码器,用于在文本和字节之间相互转换。每个编解码器都有一个名称,有的有几个别名,比如 `\texttt{utf\_8}' 别名有 `\texttt{utf8}',`\texttt{utf-8}',`\texttt{U8}'。这些名称可以传给一些函数的 \texttt{encoding} 参数。

\subsection{了解编解码问题}

编解码有个一般性的 UnicodeError 异常,报告错误时几乎都会指明具体的异常

\begin{itemize}
    \item UnicodeEncodeError:编码错误,在将字符串转换为二进制序列时错误。
    \item UnicodeDecodeError:解码错误,在将二进制序列转换为字符串时错误。
    \item SyntaxError:语法错误。
\end{itemize}

\subsubsection{处理 UnicodeEncodeError}

多数非 UTF 编解码器只能处理 Unicode 字符的一小部分子集。把文本转换成字节序列时,如果目标编码中没有定义某个字符,那就会抛出 \texttt{UnicodeEncodeError} 异常,除非把 \texttt{errors} 参数传给编码方法或函数,对错误进行特殊处理。

\subsubsection{处理 UnicodeDecodeError}

不是每一个字节都包含有效的 ASCII 字符或 UTF 码。因此,将二进制转换为文本时,如果假设是这两个编码中的一个,遇到无法转换的字节序列时会抛出 \texttt{UnicodeError}。

此外,很多陈旧的8位编码,如 `cp1252',能解码任何字节序列流而不抛出错误,例如随机噪声。这会悄无声息得得到一串无用的输出。

\subsubsection{抛出 SyntaxError}

Python 3 默认使用 UTF-8 编码源码,Python 2 则默认使用 ASCII。如果加载得 .py 模块中包含 UTF-8 之外得数据,而且没有声明编码,会得到 SyntaxError 错误。

为了修复这类问题,可以在 Python 脚本文件顶部天界一个 \texttt{coding} 注释:

\begin{python}
# coding:cp1252
\end{python}

现在 Python3 使用 UTF-8 编码,若要修正源码得陈旧编码问题,最好将其转换成 UTF-8 编码,而不是使用 \texttt{coding} 注释。

值得一提的是,Python3 原则上是允许在源码中使用非 ASCII 名称的,原书作者认为这有利有弊。以普遍理性而言,不要这样做。

\subsubsection{如何找出字节序列的编码}

除非有人告诉你字节序列的编码,不然不能找到正确的编码方式。

仅能根据某个字节码出现的频率猜测字节序列的编码,Python 有一个库:Chardet 可以对文件使用这种方式猜测编码方式。

\subsubsection{BOM:有用的鬼符}

观察下面例子:
\begin{python}
'El'.encode('utf_16')   # b'\xff\xfeE\x00l\x00'
\end{python}

可以发现使用 \texttt{utf\_16} 编码的序列开头有几个格外的字节 \texttt{b'\textbackslash xff \textbackslash xfe'}。这是 BOM,即字节序标记(byte-order mark),指明编码时使用 Intel CPU 的小字节序。在大字节序 CPU 中,编码顺序相反。

UTF-16 有两个变种:UTF-16LE 显示指明使用小写字符, UTF-16BE,显式指明使用大写字符。如果使用这两个变种,则不会生成 BOM\footnote{BOM的概念多为一些协议规则,请根据实际考虑,这里不多做解释。}。

\subsection{处理文本文件}

处理文本一般采用``三明治''方法,即在读入文件时解码输入的字符串,在输出文件时将字符串编码成字节序列。而在处理过程中,只处理字符文本。

Python3 便是采用的这种方法,内置的 \texttt{open} 函数会在读取文件时作必要的解码,在写入文件时进行编码,这样我们就只需要关心字符串的处理\footnote{Python2.7 用户需要使用 io.open() 函数才能在读写文件时自动解码编码}。

可以见得,处理文本文件十分简单。但是依赖默认编码,往往会遇到问题。Python3 在对文件进行读写时,会默认使用操作系统的编码方式,例如国人用的 Window10 系统默认编码为 GBK,这时候如果读取一个 UTF-8 编码的方式则会出现乱码。为了解决这类问题,在涉及到编码解码的函数中都应该传入 \texttt{encoding} 参数。

如果没有设置 \texttt{encoding} 的值,默认值由 \texttt{local.getpreferredencoding()} 提供。

\subsection{为了正确比较而规范化 Unicode 字符串}

Unicode 有组合字符(变音符号和附加到前一个字符上的记号,打印时作为一个整体),处理起来比较复杂。

\begin{python}
s1 = 'café'
s2 = 'cafe\u0301'
s1 == s2            # False
\end{python}

上述代码中 \texttt{s1} 和 \texttt{s2} 表达的字符是一致的,在 Unicode 标准中,`é' 和 `e\textbackslash u0303' 这样的序列叫做 `标准等价物',应用程序应该把它们视作相同的字符。但是 Python 看到的是码位的序列,因此判定二者不同。

这个问题的解决方案是使用 \texttt{unicodedata.normalize} 函数提供的 Unicode 规范化。这个函数的第一个参数是四个字符串中的一个,最常用的两个如下:

\begin{itemize}
    \item NFC(Normalization Form C)\\
    使用最少的码位构成等价的字符串。是 W3C 推荐的规范化形式。
    \item NFD \\
    把组合字符分解成基字符和单独的组合字符。
\end{itemize}

西式键盘默认是 NFC 形式,但安全起见最好使用 \texttt{normalize('NFC',user\_text)} 清洗字符串。

还有两个并不常用,一般只在特殊情况使用,这里仅列出不做解释:

\begin{itemize}
    \item NFKC \\
    NFC 基础上的严格的规范化格式,多用在一些考虑兼容性的特殊字符处理场合。
    \item NFKD \\
    NFD 基础上的严格的规范化格式,多用在一些考虑兼容性的特殊字符处理场合。
\end{itemize}

\subsubsection{大小写折叠}

大小写折叠其实就是把所有文本变成小写,再做转换。这个功能由 \texttt{str.casefold()} 方法支持。

str.casefold() 方法与 str.lower() 方法处理结果几乎一样,自 Python3.4 起,它们处理得到不同结果的有 116 个码位,而 Unicode 6.3 则有 110122 个字符,占比 0.11\%,处为特殊情况,否则无需在意。

\subsubsection{规范化文本匹配实用函数}

对大多数应用来说,NFC是最好的规范化形式。不区分大小写的比较应该使用str.casefold()。
如果要处理多语言文本,应该有\texttt{nfc\_equal}和\texttt{fold\_equal}函数。

\lstinputlisting[language=Python]{../../Scripts/DataStruct/4-13.py}

\subsubsection{极端``规范化'':去掉变音符号}

去掉重音符并不是正确的规范化方法,但现实生活往往需要这样做。

\lstinputlisting[language=Python]{../../Scripts/DataStruct/4-14.py}

原文还有一些特殊字符的处理方法,但是本人并不会德语/俄文,作为中文用户也用不到这些操作,故省略,这些内容可参阅原书 104 页。


\newpage