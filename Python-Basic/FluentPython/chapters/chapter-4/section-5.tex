\section{继承的优缺点}
\subsection{子类化内置类型很麻烦}

Python 允许内置类型子类化,但有个重要的注意事项:内置类型不会调用用户定义的类覆盖的特殊方法。

\lstinputlisting[language=Python]{../../Scripts/Class/12-1.py}

在上述代码中,只有直接调用 \texttt{\_\_setitem\_\_} 的 \texttt{dd['two'] = 2} 代码返回了我们所期望的指。而其他间接调用 \texttt{\_\_setitem\_\_} 的方法都忽略了我们覆盖的 \texttt{\_\_setitem\_\_} 方法。不仅如此,内置类型的方法调用的其他类方法,如果被覆盖了,也不会被调用。

原生类型的这种行为违背了面向对象编程的一个基本原则:始终应该从实例 (\texttt{self}) 所属的类开始搜索方法,即使在超类实现的类中调用也是如此。在这种糟糕的局面中, \texttt{\_\_missing\_\_} 方法却能按预期方法工作,不过这只是特例。

直接子类化内置类型容易出错,因为内置类型的方法通常会忽略用户覆盖的方法。用户自定义的类应该继承自 \texttt{collections} 模块中的类,这些类做了特殊设计,因此易于扩展。

下面是一个继承自 \texttt{collections.UserDict} 的例子:

\lstinputlisting[language=Python]{../../Scripts/Class/12-3.py}

\subsection{多重继承和方法解析顺序}

任何实现多重继承的语句都要处理潜在的命名冲突,这种冲突由不相关的祖先类实现同名方法引起。我们看下面这个例子。

\lstinputlisting[language=Python]{../../Scripts/Class/12-4.py}

下面我们在 \texttt{D} 的实例上调用 \texttt{d.pong()} 方法:

\begin{python}
>>> d = D()
>>> d.pong()    # 直接调用 pong 运行的是第一个继承的 B 版本。
pong:<diamond.D object at 0x......>
>>> C.pong(d)   # 超类中的方法都可以直接调用,此时要把实例作为显式参数传入。
PONG:<diamond.D object at 0x......>
\end{python}

Python 能区分 \texttt{d.pong()} 调用的是哪个方法,是因为 Python 会按照特定的顺序遍历继承图。这个顺序叫方法解析顺序。类都有一个名为 \texttt{\_\_mro\_\_} 的属性,它的值是一个元组,按照方法解析顺序列出各个超类,从当前类一直向上,直到 \texttt{object} 类。D 类的 \texttt{\_\_mro\_\_} 属性如下:

\begin{python}
>>> D.__mro__
(<class 'diamond.D'>, <class 'diamond.B'>, <class 'diamond.C'>, <class 'diamond.A'>, <class 'object'>)
\end{python}

若想把方法调用委托给超类,推荐的方式是使用内置的 \texttt{super()} 函数。有时可能要绕过方法解析顺序,直接调用某个超类的方法,这样做有时更方便。例如 \texttt{D.ping} 方法可以这样写:

\begin{python}
def ping(self):
    A.ping(self)    # 不是 super().ping()
    print('posy-ping:', self)
\end{python}

注意,直接在类上调用实例方法时,必须显式传入 \texttt{self} 参数,因为这样访问的是未绑定方法。但是最好还是使用 \texttt{super()} 方法,这样既安全,又不易过时。尤其是在调用框架或不受自己控制的类层次结构时。

下面来看在 \texttt{D} 上调用 \texttt{pingpong} 方法得到的结果。

\begin{python}
>>> d = D()
>>> d.pingpong()
post-ping: <__main__.D object at 0x00000207D6E80148>    
ping: <__main__.D object at 0x00000207D6E80148>     # 调用超类 A 中的 ping() 方法
pong: <__main__.D object at 0x00000207D6E80148>     # 自己没有,向上找到 B 的 pong() 方法
pong: <__main__.D object at 0x00000207D6E80148>     # 向上找到 B 的 pong() 方法
PONG: <__main__.D object at 0x00000207D6E80148>     # 忽略 __mro__,直接调用 C 中的 pong() 方法
\end{python}

方法解析顺序不仅考虑继承图,还考虑子类声明中列出超类的顺序。也就是说 \texttt{class D(B, C)} 会先搜索 B 再搜索 C。在内部实现中,方法解析顺序使用 C3 算法,不过除非大量使用复杂的多继承,否则无需了解。

\subsection{处理多重继承}

在日常使用中,虽然多继承没有缺陷,但是应该尽量避免使用多继承。多继承会增加复杂度,往往会导致难以理解的错误。下面是避免多继承混乱的一些建议。\footnote{这里跳过了原书 12.3 节,这节讲解的是 Tkinter 的多继承实例}

\begin{enumerate}
    \item \textit{把接口继承可实现继承区分开} 
    
    使用多重继承时,一定要明确一开始为什么创建子类:
    \begin{itemize}
        \item 继承接口:创建子类时,实现 ``是什么'' 关系。
        \item 继承实现:通过重用避免代码重复。
    \end{itemize}

    \item \textit{使用抽象基类显式表示接口} 
    
    如果类的作用是定义接口,应该明确把它定义为抽象基类。

    \item \textit{通过混入重用代码} 
    
    如果一个类的作用是为了多个不相关的子类提供方法实现,从而实现重用,但不体现 ``是什么'' 关系,应该把那个类明确定义为混入类。从概念上讲,混入不定义新类型,只是打包方法,便于重用。混入类绝对不能实例化,而且具体类不能只继承混入类。混入类应该提供某方面的特定行为,只实现少量关系非常密切的方法。

    \item \textit{在名称中明确指明混入} 
    
    Python 并没有把类声明为混入的正规方式,所以强烈建议在名称中加入 \texttt{...Mixin} 后缀表示混入类。在 UML 中使用 \textit{<<mixin>>} 标记。

    \item \textit{抽象基类可以作为混入,反过来则不成立} 
    
    抽象基类可以实现具体方法,因此可以作为混入使用。不过,抽象基类会定义类型,而混入做不到。此外,抽象基类可以作为其他类的唯一基类,而混入绝不能作为唯一的超类,除非继承另一个更具体的混入(实际上很少这样用)。

    \item \textit{不要子类化多个基类} 
    
    具体类可以没有,或者最多只有一个具体超类。也就是说,具体类的超类中除了这一个具体超类之外,其余的都是抽象基类或混入。

    \item \textit{为用户提供聚合类} 
    
    如果抽象基类或混入的组合对客户代码非常有用,那就提供一个类,使用易于理解的方式把它们结合起来。

    \item \textit{优先使用对象组合,而不是类继承} 
    
    优先使用组合能让设计更灵活。子类化是一种紧耦合,而且较高的继承树容易倒。
\end{enumerate}

\newpage