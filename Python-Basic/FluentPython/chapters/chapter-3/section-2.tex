\section{使用一等函数实现设计模式}
\subsection{案例分析:重构``策略''模式}

如果合理利用作为一等对象的函数,某些设计模式可以简化。

\subsubsection{经典的 ``策略'' 模式}

\textit{定义一系列算法,把它们一一封装起来,并且使它们可以相互替换。本模式使得算法可以独立于使用它的客户而变化。}  

\hfill —— <<设计模式:可复用面向对象软件的基础>>

图 \ref{fig:``策略''设计模式} 中的 UML 类图指出的 ``策略'' 模式对类的编排。

\begin{figure}[H]
    \centering
    \begin{tikzpicture}[scale = 1]
        \node (order) at (1,0) [draw,font=\small,thick] {
            \begin{tabular}{c}
                \textbf{Order} \\
                \midrule
                total() \\
                due() \\
            \end{tabular}
        };

        \node at (1,1.5) {上下文};

        \node (promotion) at (8,0) [minimum width=3cm,draw,font=\small,thick] {
            \begin{tabular}{c}
                \textbf{Promotion} \\
                \midrule
                \textit{discount()} \\
            \end{tabular}
        };

        \node at (8,1.5) {策略};

        \node (fidelityPromp) at (3,-4) [minimum width=3cm,draw,font=\small,thick] {
            \begin{tabular}{c}
                \textbf{FidelityPromp} \\
                \midrule
                discount() \\
            \end{tabular}
        };

        \node (bulkItemPromo) at (8,-4) [minimum width=3cm,draw,font=\small,thick] {
            \begin{tabular}{c}
                \textbf{BulkItemPromo} \\
                \midrule
                discount() \\
            \end{tabular}
        };

        \node (largeOrderPromo) at (13,-4) [minimum width=3cm,draw,font=\small,thick] {
            \begin{tabular}{c}
                \textbf{LargeOrderPromo} \\
                \midrule
                discount() \\
            \end{tabular}
        };

        \node at (9,-1.5) {具体策略};

        \begin{scope}
            \draw [thick,{Diamond[open,length=0.5cm]}-{Latex[round,length=0.5cm]}] (order) -- (promotion);
            \draw [thick,-{Latex[round,open,length=0.5cm]}] (fidelityPromp) -- ++ (0,2) -| (promotion);
            \draw [thick,-{Latex[round,open,length=0.5cm]}] (largeOrderPromo) -- ++ (0,2) -| (promotion);
            \draw [thick,-{Latex[round,open,length=0.5cm]}] (bulkItemPromo) --  (promotion);
        \end{scope}
    \end{tikzpicture}
    \caption{``策略''设计模式}
    \label{fig:``策略''设计模式}
\end{figure}

电商领域有个功能可以明显使用 ``策略'' 模式,即根据客户的属性或订单中的商品计算折扣。假定一个网店有以下折扣:
\begin{itemize}
    \item 有 1000 积分或以上顾客,每个折扣享 5\% 折扣。
    \item 同一个订单中,单个商品数量达到 20 个或以上,享 10\% 折扣。
    \item 订单中不同商品达到 10 个或以上,享 7\% 折扣。
    \item 一个订单仅能享受一个折扣
\end{itemize}

结合图 \ref{fig:``策略''设计模式},各个组成部分意义如下:
\begin{itemize}
    \item \textit{上下文} \\
    把一些计算委托给实现不同算法的可互换组件,它提供服务。在这个例子中,上下文是 \texttt{Order},它会根据不同的算法计算促销折扣。
    \item \textit{策略} \\
    实现不同算法的组件共同的接口。
    \item \textit{具体策略} \\
    ``策略''的具体子类。
\end{itemize}

例6-1实现了这个电商折扣中的方案。在这个示例中,实例化订单之前,系统会以某种方法选择一种促销策略,然后把它传给 \texttt{Order} 构造方法。具体怎么选策略,不在这个模式的职责范围内。

\lstinputlisting[language=Python]{../../Scripts/Function/6-1.py}

在上面这个例子中,把 \texttt{Promotion} 定义为抽象基类(ABC),这么做是为了使用 @abstractmethod 修饰器,从而明确表明所用的模式。

\subsubsection{使用函数实现``策略''模式}

在实例6-1中,每个具体的策略都是一个类,而且只定义了一个方法 \texttt{discount}。此外,策略实例没有状态(即实例属性)。它们看起来更像普通的函数,下面对它们进行重构\footnote{为减少代码量,不再重复调用实例,仅给出类和函数},把具体的策略换成了普通的函数。

\lstinputlisting[language=Python]{../../Scripts/Function/6-3.py}

使用参数不仅减少了代码量,而且调用函数比实例化类更简单。

\subsubsection{选择最佳策略:简单的方式}

下面我们定义一个函数 \texttt{best\_promo} 来选者折扣最大的方案。

\begin{python}
promos = [fidelityPromp,bulk_item_promo,large_order_promo]

def best_promo(order):
    '''选择最佳折扣'''
    return max(promo(order) for promo in promos)
\end{python}

在习惯了一等函数后,自然而然会像上述代码一样构建数据结构存储函数,并用生成器表达式使用函数。

这样定义最佳方案选择函数简单且易于阅读,但有些复杂情形下并不适用。比如要加入新的折扣方案,这时候就必须重写 \texttt{best\_promo} 函数。

\subsubsection{找出模块中的全部策略}

在 Python 中,模块也是一等对象,而且标准库提供了几个处理模块的函数。Python 文档是这样说明内置函数 \texttt{globals} 的。

\begin{itemize}
    \item \texttt{globals()} \\
    返回一个字典,表示当前的全局符号表。这个符号表始终指向当前模块(对函数或方法来说,是指定义它们的模块,而不是调用它们的模块)。
\end{itemize}

下面例子使用 \texttt{globals} 函数帮助 \texttt{best\_promo} 自动找到其他可用的 \texttt{*\_promo} 函数,过程有点曲折。

\begin{python}
promos = [globals()[name] for name in globals() # 迭代 globals() 返回字典中各个 name
          if name.endswith('_promo')
          and name != 'best_promo']             # 对各个 name 进行赛选

def best_promo(order):
    '''选择最佳折扣'''
    return max(promo(order) for promo in promos)    # 内部代码没有变化
\end{python}

收集所有可用促销的另一种方法是,在一个单独的模块中保存所有策略函数,把 \texttt{best\_promo} 排除在外。

\begin{python}
promos = [func for name,func in inspect.getmembers(promotions, inspect.isfunction)]

def best_promo(order):
    '''选择最佳折扣'''
    return max(promo(order) for promo in promos)    # 内部代码没有变化
\end{python}

在上面代码中,最大的变化是内省名为 \texttt{promotions} 的独立模块,构建策略函数列表。注意要导入 \texttt{promotions} 模块。\texttt{inspect.getmembers} 函数用于获取对象的属性,第二个参数是可选的判断条件(一个布尔函数)。

不管怎么明明策略函数,上例都可用。唯一重要的是, \texttt{promotions} 模块只能包含计算订单折扣的函数。如果有人在 \texttt{promotions} 模块中使用不同的签名定义函数,那么 \texttt{best\_promo} 函数常识将其应用到订单上时会出错。

动态收集促销折扣函数的一种更为显示的方法是使用装饰器,将在下一节讨论。

\subsection{``命令''模式}

``命令''设计模式将函数作为参数传递而简化。

\begin{figure}[H]
    \centering
    \begin{tikzpicture}[scale = 1]
        \node (application) at (0,0) [draw,font=\footnotesize,thick] {
            \begin{tabular}{c}
                \textbf{Application} \\
            \end{tabular}
        };

        \node at (0,1) {客户端};
        \node at (1,-1) {接收者};
        
        \node (menu) at (3,0) [draw,font=\footnotesize,thick] {
            \begin{tabular}{c}
                \textbf{Menu} \\
            \end{tabular}
        };

        \node at (3,1) {调用者};

        \node (command) at (8,0) [minimum width=3cm,draw,font=\footnotesize,thick] {
            \begin{tabular}{c}
                \textbf{Command} \\
                \midrule
                \textit{execute()} \\
            \end{tabular}
        };

        \node at (8,1.5) {命令};

        \node (openCommand) at (3,-3.5) [minimum width=3cm,draw,font=\footnotesize,thick] {
            \begin{tabular}{c}
                \textbf{OpenCommand} \\
                \midrule
                execute() \\
            \end{tabular}
        };

        \node (pasteCommand) at (8,-3.5) [minimum width=3cm,draw,font=\footnotesize,thick] {
            \begin{tabular}{c}
                \textbf{PasteCommand} \\
                \midrule
                execute() \\
            \end{tabular}
        };

        \node (macroCommand) at (13,-3.5) [minimum width=3cm,draw,font=\footnotesize,thick] {
            \begin{tabular}{c}
                \textbf{MacroCommand} \\
                \midrule
                execute() \\
            \end{tabular}
        };

        \node at (9,-1.5) {具体命令};

        \node (document) at (8,-6) [minimum width=3cm,draw,font=\footnotesize,thick] {
            \begin{tabular}{c}
                \textbf{Document} \\
                \midrule
                insert\_text() \\
            \end{tabular}
        };

        \node at (5.5,-6) {接收者};

        \begin{scope}
            \draw [thick,{Diamond[open,length=0.4cm]}-{Latex[round,length=0.4cm]}] (menu) -- (command);
            \draw [thick,-{Latex[round,open,length=0.4cm]}] (openCommand) -- ++ (0,1.5) -| (command);
            \draw [thick,-{Latex[round,open,length=0.4cm]}] (macroCommand) -- ++ (0,1.5) -| (command);
            \draw [thick,-{Latex[round,open,length=0.4cm]}] (pasteCommand) --  (command);
            \draw [thick,-{Latex[round,length=0.4cm]}] (pasteCommand) -- (document);
            \draw [thick,{Diamond[open,length=0.4cm]}-{Latex[round,length=0.4cm]}] (macroCommand) -- ++(2.5,0) |-  (command);
            \draw [thick,-{Latex[round,length=0.4cm]}] (openCommand) -| (application);
        \end{scope}
    \end{tikzpicture}
    \caption{``命令''设计模式}
    \label{fig:``命令''设计模式}
\end{figure}

``命令''模式的目的是解耦调用操作的对象(调用者)和提供实现的对象(接收者)。这种模式的做法是:在两者之间放一个 \texttt{Command} 对象,让它实现只有一个方法(\texttt{execute})的接口,调用接收者中的方法执行所需的操作。这样,调用者无需了解接收者的接口,而且不同的接收者可以适应不同的 \texttt{Command} 子类。调用者有一个具体的命令,通过调用 \texttt{execute} 方法执行。

我们可以不为调用者提供 \texttt{Command} 实例,而是给它一个函数。此时,调用者不用调用 \texttt{command.execute()} 而是直接调用 \texttt{command()} 即可。\texttt{MacroCommand} 可以实现成定义了 \texttt{\_\_call\_\_} 方法的类。

\begin{python}
class MacroCommand:
    '''一个执行一组命令的命令'''
    def __init__(self,commands):
        self.commands = list(commands)

    def __call__(self):
        for command in self.commands:
            command()
\end{python}

\newpage