\chapter{控制流程}
\section{可迭代对象,迭代器和生成器}

Python 中,所有的生成器都是迭代器,因为生成器完全实现了迭代器接口。并且,大多数情况下,迭代器和生成器都被视作同一概念。

在 Python3 中,生成器具有广泛的作用。即使是内置的 \texttt{range()} 函数也返回一个类似生成器的对象,而以前则返回完整的列表。如果一定要返回列表,那么必须明确指出(如:\texttt{list(range(10))})。

在 Python 中,所有集合都是可迭代的。在 Python 语言内部,迭代器用于支持:
\begin{itemize}
    \item \texttt{for} 循环。
    \item 构建和扩展集合类型
    \item 逐行遍历文本文件
    \item 列表推导,字典推导和集合推导
    \item 元组拆包
    \item 调用函数时,使用 * 拆包实参
\end{itemize}

\subsection{\texttt{Sentence} 类第1版:单词序列}

我们要实现一个 \texttt{Sentence} 类,以此打开探索可迭代对象的旅程。我们向这个类的构造函数传入包含一些文本的字符串,然后可以逐个单词迭代。第1版要实现序列协议:

\lstinputlisting[language=Python]{../../Scripts/ControlFlow/14-1.py}

\noindent\textbf{序列可迭代的原因: \texttt{iter} 函数}

解释器需要迭代对象 \texttt{x} 时,会自动调用 \texttt{iter(x)}。内置的 \texttt{iter} 函数有以下作用:

\begin{enumerate}
    \item 检查对象是否实现了 \texttt{\_\_iter\_\_} 内置函数,如果实现了则调用,获取一个迭代器。
    \item 如果没有 \texttt{\_\_iter\_\_} 方法,但是有 \texttt{\_\_getitem\_\_} 方法,则创建一个迭代器,尝试按顺序(索引0开始)获取元素。
    \item 如果尝试失败,则抛出错误:\texttt{TypeError}:``xx object is not iterable''。
\end{enumerate}

任何序列类型都可迭代,是因为它们都实现了 \texttt{\_\_getitem\_\_} 内置方法。其实,标准的序列都实现了 \texttt{\_\_iter\_\_} 内置方法,因此我们也应该这么做。之所以对 \texttt{\_\_getitem\_\_} 方法作特殊处理,是为了向后兼容,未来可能不会再这么做。
 
在鸭子类型中,只要实现特殊的 \texttt{\_\_iter\_\_} 方法,或者实现 \texttt{\_\_getitem\_\_} 方法且 \texttt{\_\_getitem\_\_} 方法的参数是从 0 开始的整数,就可以认为对象是可迭代的。

在白鹅类型理论中,可迭代对象的定义简单一些,不过没那么灵活:如果实现了 \texttt{\_\_iter\_\_} 方法,那么就认为对象是可迭代的。此时,不需要创建子类,也不用注册,因为 abc.Iterable 类实现了 \texttt{\_\_subclasshook\_\_} 方法,下面举个例子:

\begin{python}
>>> class Foo:
...     def __iter__(self):
...         pass
>>> from collections import abc
>>> issubclass(Foo, abc.Iterable)
True 
>>> f = Foo()
>>> isinstance(f, abc.Iterable)
True
\end{python}

不过要注意,虽然前面定义的 \texttt{Sentence} 类是可迭代的,但却无法通过 \texttt{issubclass(Sentence, abc.Iterable)} 测试。

从 Python3.4 开始,检查对象 \texttt{x} 是否迭代,最准确的方法是调用 \texttt{iter(x)} 函数,如果不可迭代,再处理 \texttt{TypeError} 异常。遮蔽 \texttt{isinstance(x,abc.Iterable)} 更为准确,因为 \texttt{iter(x)} 会考虑到遗留的 \texttt{\_\_getitem\_\_} 方法。

\subsection{可迭代对象与迭代器的对比}

两者之间的关系:Python 从可迭代对象中获取迭代器。

下面我们看一个简单的例子:字符串是可迭代对象,背后是有迭代器的,只是我们看不到:
\begin{python}
>>> s = 'ABC'
>>> for char in s:
...     print(char)
\end{python}

如果没有 \texttt{for} 语句,不自动构建迭代器,我们用 \texttt{while} 语句得这样写:

\begin{python}
>>> s = 'ABC'
>>> it = iter(s)    # 使用可迭代对象构建迭代器
>>> while(True):
...     try:
...         print(next(it))     # 不断获取下一个字符
...     except StopIteration:   # 没有字符了,报错
...         del it              # 释放迭代器对象
...         break               # 退出循环
\end{python}

\texttt{StopIteration} 异常表明迭代器到头了。Python 语言内部会处理 \texttt{for} 循环和其他迭代上下文中的 \texttt{StopIteration} 异常。

标准得迭代器接口有两个方法。
\begin{itemize}
    \item \texttt{\_\_next\_\_}
    
    返回下一个可用的元素,如果没有,抛出 \texttt{StopIteration} 异常。

    \item \texttt{\_\_iter\_\_}
    
     返回 \texttt{self} ,以便在应该使用可迭代对象的地方使用迭代器,例如 \texttt{for} 循环中。
\end{itemize}

这个接口在 collections.abc.Iterable 抽象基类中制定。

因为迭代器只需要 \_\_next\_\_ 和 \_\_iter\_\_ 两个方法,所以除了调用 \texttt{next()} 方法,以及捕获 \texttt{StopIteration} 异常之外,没有办法检查是否还有遗留的元素。此外,也没有办法 ``还原'' 迭代器。如果想再次迭代,就要调用 \texttt{iter(...)} ,传入之前构建迭代器得可迭代对象。传入迭代器本身没有用,因为 \texttt{Iterator.\_\_iter\_\_} 方法的实现方式是返回实例本身,所以传入迭代器无法还原已经耗尽的迭代器。
 
\begin{python}
# abc.Iterator 类
class Iterator(Iterable):   # Iterable 只有 __iter__ 一个内置函数
    __slots__ = ()

    @abstractmethod
    def __next__(self):
        raise StopIteration

    def __iter__(self):
        return self

    @classmethod
    def __subclasshook__(cls, C):
        if cls is Iterator:
            if (any("__next__" in B.__dict__ for B in C.__mro__) and
                any("__iter__" in B.__dict__ for B in C.__mro__)):
                return True
            return NotImplemented
\end{python}

现在我们总结一下可迭代对象与迭代器:
\begin{itemize}
    \item \textit{可迭代对象}
    
    使用 \texttt{iter} 内置函数可以获取迭代器的回想。如果对象实现了能返回迭代器的 \texttt{\_\_iter\_\_} 方法,那么对象是可迭代。序列都可以迭代;实现了 \texttt{\_\_getitem\_\_} 方法,而且其参数是从零开始的索引,这种对象也可以迭代。

    可迭代对象一般通过 \texttt{iter(x)} 进行检验。

    \item \textit{迭代器}
    
    实现了无参数的 \texttt{\_\_next\_\_} 方法,返回序列的下一个元素;如果没有元素了,那么抛出 \texttt{StopIteration} 异常。Python 中的迭代器还实现了 \texttt{\_\_iter\_\_} 方法,因此迭代器也可以迭代。

    迭代器对象一般通过 \texttt{isinstance(x, abc.Iterator)} 进行检验。
\end{itemize}

\subsection{\texttt{Sentence} 类第2版:典型的迭代器}

这一版的对象实现典型的迭代器涉及模型,但并不符合 Python 习惯做法,后面重构会说明原因。这一版能明确可迭代的集合和迭代器对象之间的关系。

\lstinputlisting[language=Python]{../../Scripts/ControlFlow/14-4.py}

对于上面示例的 \texttt{SentenceIterator} 来说,没有必要实现 \texttt{\_\_iter\_\_} 方法,因为几乎不会使用到该方法,实现的目的是为了满足迭代器协议,而且这么做能让迭代器通过 \texttt{issubclass(SentenceIterator, abc.Iterator)} 测试。如果让 \texttt{SentenceIterator} 类继承 \texttt{abc.Iterator} ,那么它会继承对应的 \texttt{\_\_iter\_\_} 方法。

\noindent\textbf{把 Sentence 编程迭代器:坏主意}

构建可迭代对象与迭代器时经常会出现错误,我们必须注意,迭代器可以迭代,但是可迭代对象不是迭代器。原因是可迭代对象的有个 \texttt{\_\_iter\_\_} 方法,每次都实例化一个新的迭代器;而迭代器要实现 \texttt{\_\_next\_\_} 方法,返回单个元素,此外还要实现 \texttt{\_\_iter\_\_} 方法,返回迭代器本身。

如果我们在 \texttt{Sentence} 类中实现 \texttt{\_\_next\_\_} 方法,让 \texttt{Sentence} 实例既是可迭代对象,也是自身的迭代器。可是这种方法是十分糟糕的,绝对不要这样使用。其主要原因是支持多种便利,即必须从同一个迭代实例中获取多个独立的迭代器,而且各个迭代器要维护自身的内部状态。

\subsection{\texttt{Sentence} 类第3版:生成器函数}

实现相同功能,但却符合Python习惯的方式是,用生成器函数代替 \texttt{SentenceIterator} 类。

\lstinputlisting[language=Python]{../../Scripts/ControlFlow/14-5.py}

在上述示例中,迭代器其实是生成器对象,每次调用 \texttt{\_\_iter\_\_} 方法都会自动创建,因为这里的 \texttt{\_\_iter\_\_} 方法是生成器函数。

\noindent\textbf{生成器函数的工作原理}

只要 Python 函数的定义体中有 \texttt{yield} 关键字,该函数就是生成器函数。调用生成器函数时,会返回一个生成器 对象。也就是说,生成器函数是生成器工厂。

生成器函数会创建一个生成器对象,包装生成器函数的定义体。把生成器传给 \texttt{next(...)} 函数时,生成器函数会向前,执行函数定义体中的下一个 \texttt{yield} 语句,返回产出的值,并在函数定义体的当前位置暂停。最终,函数的定义体返回时,外层的生成器对象会抛出 \texttt{StopIteration} 异常 —— 这一点与迭代器协议一致。

注意,生成器对象不需要 \texttt{return} 语句,如果有并遇到了 \texttt{return} 则会抛出 \texttt{StopIteration} 错误。

下面函数将更佳清楚地说明生成器函数工作原理。

\begin{python}
>>> def gen_AB():       # 定义生成器函数与普通函数无异
...     print('start')
...     yield 'A'           # 第一次调用 __next__ 时,停在此处
...     print('continue')
...     yield 'B'           # 第二次调用停在此处
...     print('end.')       # 第三次调用,打印 end. 并抛出 StopIteration 异常
... 
>>> for c in gen_AB():
...     print('-->', c)
... 
start 
---> A 
continue 
---> B 
end. 
>>>         # for 循环捕捉了 StopIteration 异常,因此没有报错
\end{python}
 
\subsection{\texttt{Sentence} 类第4版:惰性实现}

惰性在编程语言和 API 中被认为是好的特性。惰性实现是指尽可能延后生成值。这样做能节省内存,而且获取还可以避免做无用的处理。

设计 \texttt{Iterator} 接口时考虑到了惰性: \texttt{next(my\_iterator)} 一次生成一个元素。目前实现的几版 \texttt{Sentence} 类都不具有惰性,因为 \texttt{\_\_init\_\_} 方法急迫地构建好了文本中的单词列表,然后将其绑定到 \texttt{self.words} 属性上。这样就得处理整个文本,降低了效率。

\texttt{re.finditer} 函数是 \texttt{re.findall} 函数的惰性版本,返回的不是列表,而是一个生成器,按需生成 \texttt{re.MatchObject} 实例,这将节省大量内存。

\lstinputlisting[language=Python]{../../Scripts/ControlFlow/14-7.py}

\subsection{\texttt{Sentence} 类第5版:生成器表达式}

简单的生成器函数,可以替换成生成器表达式。

生成器表达式可以理解为列表推导的惰性版本:不会迫切地构建列表,而是返回一个生成器,按需惰性生成元素。也就是说,如果列表推导是制造列表的工程,那么生成器表达式就是制造生成器的工厂。

生成器表达式会产出生成器,由此可以构建更为精简的 \texttt{Sentence} 类:

\lstinputlisting[language=Python]{../../Scripts/ControlFlow/14-9.py}

\subsection{何时使用生成器表达式}

生成器表达式是语法糖:完全可以替换成生成器函数,不过有时生成器表达式更佳便利。

生成器表达式是创建生成器的简洁句法,这样无需先定义函数再调用。不过,生成器灵活得多,可以使用多个语句实现复杂的逻辑,也可以作为协程\footnote{后文会讲。}使用。

选择何种句法很容易判断:如果生成器表达式要分成多行写,倾向于定义生成器函数,以便提高可读性。此外生成器函数可以重用。

如果函数或构造方法只有一个参数,传入生成器表达式时不用写一对调用函数的括号,再写一对生成器表达式的阔含,只写一对就行了:

\begin{python}
    return Vector(n * scalar for n in self)
\end{python}

\subsection{等差数列生成器}

下面我们来创建一个等差数列生成类:

\lstinputlisting[language=Python]{../../Scripts/ControlFlow/14-11.py}

在最后一行,我们没有使用 self.step 不断地增加 result 而是应用了 index 获得,这样避免了处理浮点数时累积效应引起的风险。

在这个类中,我们应用生成器表达式的特性,实现了无穷数列的生成。

这个示例简单地说明了如何使用生成器函数实现特殊的 \texttt{\_\_iter\_\_} 方法。不过如果一个类只是为了构建生成器而去实现 \texttt{\_\_iter\_\_} 方法,那还不如使用生成器函数。

\lstinputlisting[language=Python]{../../Scripts/ControlFlow/14-12.py}

\noindent\textbf{使用 \texttt{itertools} 模块生成等差数列}

Python3.4 中的 \texttt{itertools} 模块提供了 19 个生成器函数,结合起来使用能实现很多有趣的用法。

例如, \texttt{itertools.count} 函数返回的生成器能生成多个数,如果不传入参数,生成从零开始的整数数列。也可以提供可选的 \texttt{start} 和 \texttt{step} 值。

\begin{python}
    >>> import itertools
    >>> gen = itertools.count(1, .5)
    >>> next(gen)
    1
    >>> next(gen)
    1.5
\end{python}

然而,像 \texttt{itertools.count} 这样的生成无穷个元素的函数从不停止。因此,如果调用 \texttt{list(count())} ,Python 会创建一个特别大的表,导致电脑崩溃。

不过, \texttt{itertools.takewhile} 函数则不同,它会生成一个使用另一个生成器的生成器,在指定的条件计算结果为 \texttt{False} 时停止。因此,可以把这两个函数结合起来使用,编写下述代码:

\begin{python}
    >>> gen = itertools.takewhile(lambda n: n<3 , itertools.count(1, .5))
    >>> list(gen)
    [1, 1.5, 2.0, 2.5]
\end{python}

下面使用这两个函数,写出有穷的等差数列函数:

\lstinputlisting[language=Python]{../../Scripts/ControlFlow/14-13.py}

注意上述函数不是生成器函数,而是生成器工厂函数。

\subsection{标准库中的生成器函数}

标准库中提供了很多生成器,下面将按功能分组,简单介绍其中的 24 个。\footnote{这里每个函数只提供了一个示例,更全面的用法请看原书P350或参考官方文档。}

第一组是用于过滤的生成器函数:从输入的可迭代对象中产出元素的子集,而且不修改元素本身。就像 \texttt{itertools.takewhile} 函数一样,下表中的大多数函数都接受一个断言参数。这个参数是个布尔函数,有一个参数,会应用到输入中的每个元素上,用于判断元素是否包含在输出中。

\begin{table}[H]
    \centering
    \caption{用于过滤的生成器函数}
    \label{table:用于过滤的生成器函数}
    \setlength{\tabcolsep}{2mm}
    \small
    \begin{tabular}{l|l|p{9cm}}
        \toprule
        \textbf{模块} & \textbf{函数} & \textbf{说明} \\
        \midrule
        \texttt{itertools} & \texttt{compress(it, selector\_it)} & 并行处理两个可迭代的对象,如果 \texttt{selector\_it} 中的元素是真值,产出 it 中对应的元素。 \\
        \midrule
        \texttt{itertools} & \texttt{dropwhile(predicate, it)} & 处理 \texttt{it} 跳过 \texttt{predicate} 的计算结果为真值的元素,然后产出剩下的各个元素。(不再进一步检查) \\
        \midrule
        (内置) & \texttt{filter(predicate, it)} & 把 \texttt{it} 中的各个元素传给 \texttt{predicate} ,如果 \texttt{predict(item)} 返回真值,那么产出对应的元素;如果 \texttt{predict} 是 \texttt{None},那么只产生真值元素。\\
        \midrule
        \texttt{itertools} & \texttt{filterfalse(predicate, it)} & 与 \texttt{filter} 函数的作用类似,不过 \texttt{predicate} \texttt{的逻辑是相反的:predicate} 返回假值时产出对应的元素 \\
        \midrule
        \texttt{itertools} & \texttt{islice(it,start,stop,step)} &  产出 \texttt{it} 的切片,作用类似于 \texttt{s[:stop]} 或 \texttt{s[start:stop:step]},不过 \texttt{it} 可以是任何可迭代的对象,而且这个函数实现的是惰性操作。 \\
        \midrule
        \texttt{itertools} & \texttt{takewhile(predicate, it)} & \texttt{predicate} 返回为真值时产出对应元素,然后立即停止,不再继续检查。 \\
        \bottomrule
    \end{tabular}
\end{table}

下面演示各个函数的用法:
\begin{python}
>>> import itertools
>>> def vowel(c):
...     return c.lower() in 'aeiou'
... 
>>> list(filter(vowel, 'Aardvark'))     # 返回真值则产生
['A','a','a']
>>> list(itertools.filterfalse(vowel,'Aardvark'))     # 返回假值则产生
['r','d','v','r','k']
>>> list(itertools.dropwhile(vowel,'Aardvark'))     # 跳过前面的真值
['r','d','v','a','r','k']
>>> list(itertools.takewhile(vowel,'Aardvark'))     # 只产生前面的真值
['A','a']
>>> list(itertools.compress('Aardvark',(1,0,1,1,0,1)))  # 产生对应为真的值
['A','r','d','a']
>>> list(itertools.islice('Aardvark',1,7,2))    # 类似切片操作
['a','d','a']
\end{python}

下一组用于映射的生成器函数:在输入的单个可迭代对象中的各个元素上做计算,然后返回结果。

\begin{table}[H]
    \centering
    \caption{用于映射的生成器函数}
    \label{table:用于映射的生成器函数}
    \setlength{\tabcolsep}{2mm}
    \small
    \begin{tabular}{l|l|p{9cm}}
        \toprule
        \textbf{模块} & \textbf{函数} & \textbf{说明} \\
        \midrule
        \texttt{itertools} & \texttt{accumulate(it,[func])} & 产出累计的总和:如果提供了 \texttt{func},那么把前两个元素传给它,然后把结果和下一个元素运算,以此类推。 \\
        \midrule
        (内置) & \texttt{enumerate(iterable, start=0)} & 产出由两个元素组成的元组,结构是 \texttt{(index,item)} ,其中 \texttt{index} 从 \texttt{start} 开始计数, \texttt{item} 则从 \texttt{iterable} 中取。 \\
        \midrule
        (内置) & \texttt{map(func,it1,[it2...])} & 把 \texttt{it} 中的各个元素传给 \texttt{func} 产出结果;如果传入 N 个可迭代对象,那么 \texttt{func} 必须能接受 N 个参数,而且要并行处理各个可迭代对象。 \\
        \midrule
        \texttt{itertools} & \texttt{starmap(func,it)} & 把 \texttt{it} 中的各个元素传给 \texttt{func} 产出结果;输入的可迭代对象应该产出可迭代元素 \texttt{iit},然后以 \texttt{func(*itt)} 这种形式调用 \texttt{func}。 \\
        \bottomrule
    \end{tabular}
\end{table}

\begin{python}
>>> import itertools
>>> import operator
>>> sample = [5,4,2,8,7,6,3,0,9,1]
>>> list(itertools.accumulate(sample,max))  # 产生最值
[5,5,5,8,8,8,8,8,9,9]
>>> list(enumerate('abc',1))    # 产生编号与元素
[(1,'a'),(2,'b'),(3,'c')]
>>> list(map(operator.mul, rang(11), range(11)))    # 映射运算
[0,1,4,9,16,25,36,49,64,81,100]
>>> list(itertools.starmap(operator.mul,enumerate('abc',1)))    # 拆包映射运算
['a','bb','ccc']
\end{python}

接下来一组用于合并的生成器函数,这些函数都从输入的多个可迭代对象中产出元素。
\begin{table}[H]
    \centering
    \caption{合并多个了迭代对象的生成器函数}
    \label{table:合并多个了迭代对象的生成器函数}
    \setlength{\tabcolsep}{2mm}
    \small
    \begin{tabular}{l|l|p{9cm}}
        \toprule
        \textbf{模块} & \textbf{函数} & \textbf{说明} \\
        \midrule
        \texttt{itertools} & \texttt{chain(it1,...)} & 先产出 \texttt{it1} 中的所有元素,然后产出 \texttt{it2} 中的所有元素,以此类推。 \\
        \midrule
        \texttt{itertools} & \texttt{chain.from\_iterable(it)} & 产出 \texttt{it} 生成的各个可迭代对象的元素,一个接一个,无缝衔接在一起。  \\
        \midrule
        \texttt{itertools} & \texttt{product(it1,...,repeat=1)} & 计算笛卡尔积; \texttt{repeat} 指明重复处理多少次输入的可迭代对象。  \\
        \midrule
        (内置) & \texttt{zip(it1,...)} & 并行从输入的各个可迭代对象中获取元素,产出 N 个元素组成的元组。  \\
        \midrule
        \texttt{itertools} & \tabincell{c}{\texttt{zip\_longest(it1,...,}\\\texttt{fillvalue=None)}} & 与上一个函数类似,但持续到最长的对象,空缺值使用 \texttt{fillvalue} 填充。  \\
        \bottomrule
    \end{tabular}
\end{table}

\begin{python}
>>> import itertools
>>> list(itertools.chain('ABC', range(2)))  # 依次产出元素
['A','B','C',0,1]
>>> list(itertools.chain.from_iterable(enumerate('ABC')))   # 从生成其中依次产出元素
[0,'A',1,'B',2,'C']
>>> list(zip('ABC',range(5)))   # 并行取出元素
[('A',0),('B',1),('C',2)]
>>> list(itertools.zip_longest('ABC',range(5)))     # 并行取出(最长)
[('A',0),('B',1),('C',2),(None,3),(None,4)]
>>> list(itertools.product('ABC',range(2)))     # 笛卡尔积
[('A',0),('A',1),('B',0),('B',1),('C',0),('C',1)]
\end{python}

有些生成器函数会从一个元素中产出多个值,扩展输入的可迭代对象:
\begin{table}[H]
    \centering
    \caption{把输入的各个元素扩展成多个输出元素的生成器函数}
    \label{table:把输入的各个元素扩展成多个输出元素的生成器函数}
    \setlength{\tabcolsep}{2mm}
    \small
    \begin{tabular}{l|l|p{10cm}}
        \toprule
        \textbf{模块} & \textbf{函数} & \textbf{说明} \\
        \midrule
        \texttt{itertools} & \tabincell{l}{\texttt{combinations} \\ \texttt{(it,out\_len)}} & 把 \texttt{it} 产出的 \texttt{out\_len} 个元素组合在一起,然后产出 \\ \midrule
        \texttt{itertools} & \tabincell{l}{\texttt{combinations\_with\_re} \\ \texttt{placement(it,out\_len)}} & 把 \texttt{it} 产出的 \texttt{out\_len} 个元素组合在一起,然后产出,包含相同元素的组合 \\ \midrule
        \texttt{itertools} & \texttt{count(start=0,step=1)} & 从 \texttt{start} 开始不断产出数字,按 \texttt{step} 指定的步幅增加 \\ \midrule
        \texttt{itertools} & \texttt{cycle(it)} & 从 \texttt{it} 中产出各个元素,存储各个元素的副本,然后按顺序重复不断地产出各个元素 \\ \midrule
        \texttt{itertools} & \tabincell{l}{\texttt{permutations} \\ \texttt{it,out\_len=None}} & 把 \texttt{out\_len} 个 \texttt{it} 产出的元素排列在一起,然后产出这些排列; \texttt{out\_len} 的默认值等于 \texttt{len(list(it))} \\ \midrule
        \texttt{itertools} & \texttt{repeat(item,[times])} & 重复不断地产出指定的元素,除非提供 items,指定次数。 \\
        \bottomrule
    \end{tabular}
\end{table}

演示 \texttt{count, repeat, cycle} 的用法:

\begin{python}
>>> ct = itertools.count()      # 累加
>>> next(ct), next(ct)
(0,1)
>>> cy = itertools.cycle('ABC')     # 循环
>>> list(itertools.islice(cy,7))
['A','B','C','A','B','C','A']
>>> rp = itertools.repeat(7)    # 重复
>>> next(rp), next(rp)
(7,7)
\end{python}

组合生成器(其余三个生成器函数)用法:
\begin{python}
>>> list(itertools.combinations('ABC',2))
[('A','B'),('A','C'),('B','C')]
>>> list(itertools.combinations_with_replacements('ABC',2))
[('A','A'),('A','B'),('A','C'),('B','B'),('B','C'),('C','C')]
>>> list(itertools.product('ABC', repeat=2))
[('A','A'),('A','B'),('A','C'),('B','A'),('B','B'),('B','C'),('C','A'),('C','B'),('C','C')]
\end{python}

最后一组生成器函数用于产出输入的可迭代对象中的全部元素,不过会以某种方式重新排列。

\begin{table}[H]
    \centering
    \caption{用于重新排列元素的生成器函数}
    \label{table:用于重新排列元素的生成器函数}
    \setlength{\tabcolsep}{2mm}
    \small
    \begin{tabular}{l|l|p{11cm}}
        \toprule
        \textbf{模块} & \textbf{函数} & \textbf{说明} \\
        \midrule
        \texttt{itertools} & \tabincell{l}{\texttt{groupby} \\ \texttt{(it,key=None)}} & 产出由两个元素组成的元素,形式为 \texttt{(key,group), 其中 \texttt{key} 是分组标准, \texttt{group} 是生成器,用于产出分组里的元素。}  \\ \midrule
        (内置) & \texttt{reversed(seq)} & 从后向前,倒序产出 \texttt{seq} 中的元素, \texttt{seq} 必须是序列,或者实现了 \texttt{\_\_reversed\_\_} 特殊方法的对象。 \\ \midrule
        \texttt{itertools} & \texttt{tee(it,n=2)} & 产出一个由 \texttt{n} 个生成器组成的元组,每个生成器用于单独产出输入的可迭代对象中的元素。 \\
        \bottomrule
    \end{tabular}
\end{table}

\texttt{itertools.groupby} 生成器函数的用法

\begin{python}
>>> list(itertools.groupby('LLLLAAGGG'))
[('L', <itertools._grouper object at 0x000001CF2156C7C8>), 
('A', <itertools._grouper object at 0x000001CF2156C848>), 
('G', <itertools._grouper object at 0x000001CF2156C8C8>)]
>>> animals = ['duck','eagle','rat','giraffe','bear','bat','dolphin','shark','lion']
>>> animals.sort(key=len)
>>> for length,group in itertools.groupby(animals,len):
...     print(length, '->', list(group))
3 -> ['rat', 'bat']
4 -> ['duck', 'bear', 'lion']
5 -> ['eagle', 'shark']
7 -> ['giraffe', 'dolphin']
\end{python}

\texttt{itertools.tee} 生成器函数的用法

\begin{python}
>>> list(itertools.tee('ABC'))
[<itertools._tee object at 0x000001CF21572108>, <itertools._tee object at 0x000001CF21572148>]
>>> g1, g2 = itertools.tee('ABC')
>>> next(g1)
'A'
>>> next(g2)
'A'
\end{python}

\subsection{新句法 \texttt{yield from}}

\texttt{yield from} 是 Python3.3 中新出现的句法。如果生成器函数需要产出一个生成器生成值,传统的解决方法是使用嵌套的 \texttt{for} 循环。\footnote{标准库中的 \texttt{chain} 函数是用C语言写的。}

\begin{python}
>>> def chain(*iterables):
...     for it in iterables:
...         for i in it:
...             yield i
... 
>>> s = 'ABC'
>>> t = tuple(range(3))
>>> list(chain(s,t))
['A','B','C',0,1,2]
\end{python}

\texttt{chain} 生成器函数把操作依次交给接收到的各个可迭代对象处理。为此,引入了一个新句法,如下述代码所示:

\begin{python}
>>> def chain(*iterables):
...     for it in iterables:
...         yield from it
\end{python}

可以看出, \texttt{yield from} 完全代替了内层的 \texttt{for} 循环。除了代替循环之外,\texttt{yield from} 还会创建通道,把内层生成器直接与外层生成器的客户端联系起来。\footnote{后文会有更详细的说明。}

\subsection{可迭代的归约函数}

归约函数接受一个可迭代的对象,然后返回单个结果。下表列出的规约函数其实都可以使用 \texttt{functools.reduce} 函数实现,内置是因为使用它们便于解决常见的问题。此外,对 \texttt{all} 和 a\texttt{ny} 函数来说,有一项重要的优化措施是 \texttt{reduce} 函数做不到的:这两个函数会短路。(一旦确定解决立马停止计算)

\begin{table}[H]
    \centering
    \caption{读取迭代器,返回单个值的内置函数}
    \label{table:读取迭代器,返回单个值的内置函数}
    \setlength{\tabcolsep}{2mm}
    \small
    \begin{tabular}{l|l|p{11cm}}
        \toprule
        \textbf{模块} & \textbf{函数} & \textbf{说明} \\
        \midrule
        (内置) & \texttt{all(it)} & \texttt{it} 中所有元素都为真时返回 \texttt{True},空值返回 \texttt{True} \\ \midrule
        (内置) & \texttt{any(it)} & \texttt{it} 中存在元素为真时返回 \texttt{True},空值返回 \texttt{False}  \\ \midrule
        (内置) & \tabincell{l}{\texttt{max(it,[key=,]} \\ \texttt{[default=])}} & 返回最大元素, \texttt{key} 是排序函数,如果可迭代对象为空,返回 \texttt{default} \\ \midrule
        (内置) & \tabincell{l}{\texttt{min(it,[key=,]} \\ \texttt{[default=])}} & 返回最小元素, \texttt{key} 是排序函数,如果可迭代对象为空,返回 \texttt{default}  \\ \midrule
        \texttt{functools} & \tabincell{l}{\texttt{reduce(func,it,} \\ \texttt{[initial])}} & 把前两个元素传给 \texttt{func},然后把计算结果和第三个元素传给 \texttt{func};如果提供了 \texttt{initial} 则将其作为第一个元素。  \\ \midrule
        (内置) & \texttt{sum(it,start=0)} & 计算总和,如果提供了 \texttt{start} 会把它加上。  \\
        \bottomrule
    \end{tabular}
\end{table}

还有一个内置的函数接受一个可迭代对象,返回不同的值 —— \texttt{sorted}。 \texttt{reversed} 是生成器函数,与此不同, \texttt{sorted} 会构建并返回真正的列表。

\subsection{深入分析 \texttt{iter} 函数}

如前所述,在 Python 中迭代对象 \texttt{x} 时会调用 \texttt{iter(x)}。

\texttt{iter} 函数还有一个鲜为人知的用法:传入两个参数,使用常规函数或任何可调用的对象创建迭代器。这样使用时,第一个参数必须是可调用对象,用于不断调用,产出值;第二个参数是哨符,这是个标记值,当可调用的对象返回这个值时,触发迭代器抛出 \texttt{StopIteration} 异常,而不产出哨符。

应用这个特性,我们构建一个致骰子游戏,知道掷出1点为止:
\begin{python}
>>> def d6():
...     return randint(1,6)
... 
>>> d6_iter = iter(d6,1)
>>> for roll in d6_iter:
...     print(roll)
... 
4 3 6 3
\end{python}

这里绝对不对产生1,因为1是我们的哨符。

内置函数 \texttt{iter} 的文档中有个使用的例子。这段代码逐行读取文件,直到遇到空行或者到达文件末尾位置:
\begin{python}
with open('mydata.text') as fp:
    for line in iter(fp.readline, '\n'):
        process_line(line)
\end{python}

本章到这里结束,原书 P360 页还有对数据处理方面的介绍。

\newpage