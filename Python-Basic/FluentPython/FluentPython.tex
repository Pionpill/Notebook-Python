\documentclass{PionpillNote-book}

\import{chapters/style}{tikz-style.tex}

\title{Fluent Python 笔记}
\author{
    Pionpill \footnote{笔名:北岸,电子邮件:673486387@qq.com,Github:\url{https://github.com/Pionpill}} \\
    本文档为作者学习《Fluent Python》\footnote{《Fluent Python》:Luciano Ramalho 2017年中文第一版}一书时的笔记。\\
}

\date{\today}

\begin{document}

\maketitle

\noindent\textbf{前言:}

笔者为软件工程系在校本科生,主要利用 Python 进行数据科学与机器学习使用,也有一定游戏开发经验。

<<Fluent Python>> 是  Python 学习的进阶书籍,在学习本书之前,本人已拜读过 <<Python 从入门到实践>>\footnote{俗称:蟒蛇书},<<Automate with Python>> 等书。本书更多聚焦于Python内部处理,本人在阅读本书时获得了极大的收获,惊喜程度不亚于初读蟒蛇书。

书上有大量的示例,在本文中也有给出,脚本位于 Scripts 文件下,但本人只书写了部分脚本,还有小部分本人觉得没有必要,或者是以命令行形式书写,这些并没有给出。已有的脚本推荐读者使用 VSCode 和 Python Preview 插件,便于查看脚本运行时的内部逻辑。此外,原文有大段对 Python3\footnote{原书支持到 Python3.4} 和 Python2 的对比,除非特别必要,本人不会再详述 Python2 \footnote{特指 Python2.7 ,再远古的版本不再提及}的相关内容,除非特殊说明,默认适用于 Python3 和 CPython 解释器。

本笔记不能代替原书,仅是对原书的一个总结归纳;原书大段精妙的解释均没有被记录在笔记中。本人极其推荐有一定 Python 基础的人购买原书阅读,直到截稿日期,我都认为本书是我在学习 Python 路线上阅读过最好的书籍之一。

本笔记只是对原书的马虎概括与整理,如有疑问或需求,还请购买原书;本文所在Github仓库采用 GPL v3 协议,但请勿将本文商业使用,本文引用了一些 CSDN 或其他论讨的文章,如果原作者觉得不合适,请联系本人。

\date{\today}

\tableofcontents
\thispagestyle{empty}
\newpage
\setcounter{page}{1} 

\import{chapters/chapter-1}{section-1.tex}

\import{chapters/chapter-2}{section-1.tex}
\import{chapters/chapter-2}{section-2.tex}
\import{chapters/chapter-2}{section-3.tex}

\import{chapters/chapter-3}{section-1.tex}
\import{chapters/chapter-3}{section-2.tex}
\import{chapters/chapter-3}{section-3.tex}

\import{chapters/chapter-4}{section-1.tex}
\import{chapters/chapter-4}{section-2.tex}
\import{chapters/chapter-4}{section-3.tex}
\import{chapters/chapter-4}{section-4.tex}
\import{chapters/chapter-4}{section-5.tex}
\import{chapters/chapter-4}{section-6.tex}

\import{chapters/chapter-5}{section-1.tex}
\import{chapters/chapter-5}{section-2.tex}



\end{document}

