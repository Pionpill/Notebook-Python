\documentclass{PionpillNote-book}

\import{chapters/style}{tikz.tex}

\title{机器学习笔记}
\author{
    Pionpill \footnote{笔名:北岸,电子邮件:673486387@qq.com,Github:\url{https://github.com/Pionpill}} \\
    本文档为作者学习《机器学习》\footnote{《机器学习》:周立华,清华大学出版社,2019年第二版}一书时的笔记,\\
    主要对常用机器学习方法做了笔记。\\
}

\date{\today}

\begin{document}

\maketitle

\noindent\textbf{前言:}

笔者为软件工程系在校本科生,空余时间自学了机器学习相关内容,在撰写该文档时,笔者并没有较深的统计学或数学知识,数学知识体系仅限于理工科必修的《高等数学》《概率统计》《离散数学》《线性代数》\footnote{笔者撰写时已经一年没有接触相关知识了,在撰写过程中过深的数学知识会格外备注}四本书。数据方面自学了基本的数据处理方式,包括python基础,numpy,pandas两个重要库以及一些常用数据分析工具,这方面的知识参考《利用 Python 进行数据分析》\footnote{原书英文名:《Data Analysis for Python》}一书。在模型建立方面,笔者有一定的老本,曾在2021年MCMICM\footnote{美国大学生数学建模竞赛}中获M奖,主要负责编程,数据分析,排版。以上为笔者的基本介绍与知识系统说明,供各位学习时对比参考。

笔记的书写顺序与《机器学习》一致,章节采用了序言中的分法,将 1-3 章分为机器学习基础,4-10章分为机器学习常用方法,11-16章分为机器学习进阶方法。

原书对于一些名称或公式并没有详细解读,而是跳跃性说明,因此本文对于一些专有名词,会编写脚注,若经常用到或难以理解,会附上本人收集到的认为有用的学习链接,并放在附录中。若有知识超过本科数学(理工科必修的四本书),会被标记本科超纲。此外,本文在编写同时,也在对《统计学习方法》\footnote{《统计学习方法》:李航,清华大学出版社,2019年第二版}一书做笔记,主要由于部分内容过于深奥,会定期求助于我的一些研究生学长学姐,也可能由于在补充知识体系而长期断更。

本书(《机器学习》)不同于《统计学习方法》,本书含有大段原作者的引导型举例,如果说《统计学习方法》一书的笔记涵盖了原书 10 单位的精华内容,那么本书的笔记则仅有 3 单位,在学习顺序上,也更推荐先看完本书再学习其他书籍。

由于本人非常不喜欢中式教科书式的死板,行文风格比较自由,还请谅解。

本笔记只是对原书的马虎概括与整理,如有疑问或需求,还请购买原书;本文所在Github仓库采用 GPL v3 协议,但请勿将本文商业使用。

\date{\today}

\tableofcontents
\thispagestyle{empty}
\newpage
\setcounter{page}{1} 

\import{chapters/chapter-1}{section-1.tex}
\import{chapters/chapter-1}{section-2.tex}

\import{chapters/appendix}{appendix.tex}




\end{document}

