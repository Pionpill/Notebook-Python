\chapter{附录}

\section{主要符号表}

\begin{table}[H]
    \centering
    \caption{主要符号表}
    \label{table:主要符号表}
    \setlength{\tabcolsep}{4mm}
    \begin{tabular}{c|c|c|c}
        \toprule
        \textbf{符号} & \textbf{意义} & \textbf{符号} & \textbf{意义} \\
        \midrule
        $x$ & 标量 & $\vec{x}$ & 向量 \\
        $\mathrm{x}$ & 变量集 & $\mathbf{A}$ & 矩阵 \\
        $\mathbf{I}$ & 单位阵 & $\mathcal{X}$ & 样本空间或状态空间 \\
        $\mathcal{D}$ & 概率分布 & $D$ & 数据样本(数据集) \\
        $\mathcal{H}$ & 假设空间 & $H$ & 假设集 \\
        $\mathcal{E} $ & 学习算法 & $(\cdot,\cdot,\cdot)$ & 行向量 \\
        $(\cdot;\cdot;\cdot)$ & 列向量 & $(\cdot)^T$ & 向量转置 \\
        $\{\cdots\}$ & 集合 & $|\{\cdots\}|$ & 集合中元素的个数 \\
        $||\cdot||_p$ & $L_p$ 范数,$p$缺省时为 $L_2$范数 & $P(\cdot),P(\cdot|\cdot)$ & (条件)概率质量函数 \\
        $p(\cdot),p(\cdot|\cdot)$ & (条件)概率密度函数 & $\sup(\cdot)$ & 上确界 \\
        $\mathbb{I} (\cdot)$ & 指数函数,$\cdot$ 为真假分别取1,0 & $\text{sign}(\cdot)$ & 符号函数,分别取-1,0,1 \\
        \midrule
        $\mathbb{E}_{\cdot ~D}[f(\cdot)]$ & \multicolumn{3}{c}{函数 $f(\cdot)$ 对 $\cdot$ 在分布 $\mathcal{D}$ 下的数学期望;意义明确时省略测$\mathcal{D},\cdot$} \\
        \bottomrule
    \end{tabular}
\end{table}