\chapter{机器学习基础}

\section{绪论}

周志华先生的行文风格更贴近于生活,一些术语并没有给出标准解释,而是用例子来说明。因此这章的许多解释都是本人的总结,更为标准的解释可以参考《统计学习方法》一书的笔记。

\subsection{基本术语}

\subsubsection{数据相关}

\begin{itemize}
    \item \textbf{数据集(data set)}
    
    记录的集合,记录包括属性与值,通常形式为(属性1 = 值1;属性2 = 值2;...)。
    \item \textbf{示例/样本(instance/sample)}
    
    数据集中的单条记录称为一个样本。有时整个数据集也称为一个样本\footnote{因为一般数据集不可能等于样本空间,故可看作对样本空间的采样}。
    \item \textbf{属性/特征(attribute/feature)}
    
    在某方面的表现或性质的事项,对应值为属性值/特征值。
    \item \textbf{属性空间/样本空间/输入控件}
    
    属性张成的空间(属性作笛卡尔积?)
    \item \textbf{特征向量(feature vector)}
    
    一个示例
\end{itemize}

令$D = \{\bm{x}_1,\bm{x}_2,\cdots,\bm{x}_m\}$ 表示包含 $m$ 个示例的数据集,每个示例由 $d$ 个属性描述,则每个示例 $\bm{x}_i = (x_{i1};x_{i2};\cdots;x_{id})$ 是 $d$ 维样本空间 $\mathcal{X}$ 中的一个向量。$\bm{x}_i \in \mathcal{X}$;其中 $x_{ij}$ 是第 $j$ 个属性上的取值,$d$ 称为样本\textbf{维度(dimensionality)}。

\subsubsection{训练相关}

\begin{itemize}
    \item \textbf{学习/训练(learning/training) }
    
    从数据中学习模型的过程。
    \item \textbf{假设(hypothesis)}
    
    学到的模型对应了关于数据的某种潜在的规则。
    \item \textbf{真相/真实(ground-truth)}
    
    潜在规律。
    \item \textbf{标记(label)}
    
    关于示例结果的信息。如是好瓜/不是好瓜
    \item \textbf{样例(example)}
    
    拥有了标记信息的示例。一般用 $(\bm{x}_i,y_i)$表示,其中 $y_i \in \mathcal{Y}$ 为标记。
    \item \textbf{输出空间(label space)}
    
    所有标记的集合。
\end{itemize}

\subsubsection{模型相关}

\begin{itemize}
    \item \textbf{分类(classification)}
    
    输出空间包含的是离散值。若只涉及两个值,称为二分类(binary classification)。通常称其中一个为正类(positive class),一个为反类(negative class)。
    \item \textbf{ 回归(regression)}
    
    输出空间包含的是连值。
\end{itemize}

一般预测任务是希望通过对训练集进行学习,建立一个从输入空间到输出空间的映射:$\mathcal{X} \rightarrow \mathcal{Y}$。对于二分类任务,通常令 $\mathcal{Y} = \{-1,+1\} \text{或} {0,1}$;对多分类任务,$|\mathcal{Y}|>2$;对回归任务,$\mathcal{Y} = \mathbb{R}(\text{实数集})$。

\begin{itemize}
    \item \textbf{聚类(clustering)}
    
    将训练集中的数据分成若干组,每组称为一个簇(cluster)。
\end{itemize}

根据训练数据是否拥有标记信息,学习任务大致可分为两大类:"监督学习"(supervised learning)和"无监督学习"(unsupervised learning)。分类和回归是前者的代表,聚类是后者的代表。

\begin{itemize}
    \item \textbf{泛化(generalization) }
    
    学得模型适用于新样本的能力。泛化能力的大小是评判模型好坏的重要标志之一。
\end{itemize}

通常,假设样本空间中全体样本服从一个未知"分布"(distribution)$\mathcal{D}$,我们获得的每个样本都是独立地从这个分布采样获得的,即\textbf{独立同分布}。

\subsection{概念学习与归纳偏好}
\subsubsection{概念学习}

\begin{itemize}
    \item \textbf{归纳(induction)}
    
    从许多个别的事物中概括出一般性概念、原则或结论的思维方法。
    \item \textbf{演绎(deduction)}
    
    由一般原理推出关于特殊情况下的结论。
\end{itemize}

归纳学习(induction learning),即从样例中学习,有侠义与广义之分,广义的归纳学习大体相当于从样例中学习,而狭义的归纳学习则要求从训练数据中学到概念,也称概念学习\footnote{概念学习不是重点,具体例子可见书 P4}。

\subsubsection{归纳偏好}

\textbf{归纳偏好(inductive bias)}指机器学习算法在学习过程中对某种类型假设的偏好。任何有效的机器学习必然有其归纳偏好。"奥卡姆剃刀"\footnote{若有多个假设与观察一致,则选最简单的那个。}是一种常用的一般性原则来引导算法确定"正确的"偏好,但也有例外\footnote{书P6-9详细解释了归纳偏好及其解决方案,但多为理解性内容,若需进一步理解请读者查看原书}。

\subsection{发展历程}

机器学习是人工智能研究发展到一定阶段的必然产物。

\subsubsection{发展历史}

\begin{itemize}
    \item \textbf{二十世纪五十年代到七十年代 (推理期)} \\
    认为只要赋予及其逻辑推理能力,机器就能具有智能。\\
    A.Newell, H.Simon 的"逻辑推理家"程序证明了《数学原理》中的定理,并获得1975年图灵奖。
    \item \textbf{二十世纪七十年代中期(知识期)} \\
    由人总结出知识交给计算机
    \item \textbf{现在} \\
    由机器自主学习
\end{itemize}

\subsubsection{主流技术}

\begin{itemize}
    \item \textbf{符号主义学习}\\
    以决策树和基于逻辑的学习为代表。直接模拟了人类对该你那进行判断的树形流程。基于逻辑学习的著名代表是归纳逻辑程序设计(Inductive Logic Programming,ILP)。\\
    决策树学习简单易用,ILP具有很强的知识表示能力,可以容易地表达出复杂数据关系。\\
    由于表示能力太强,直接导致学习时假设空间太大,复杂度极高,问题规模大时难以有效学习。二十世纪九十年代中期这方面研究逐渐低调。

    \item \textbf{连接主义学习-神经网络}\\
    二十世纪八十年代之前都不被重视,直到解出 NP 问题。\\
    连接主义最大的局限是"试错性";简单地说就是需要手动调节参数,缺乏理论指导。二十世纪九十年代中期也逐渐没落。

    \item \textbf{统计学习} \\
    代表性技术是向量机,核方法。在二十世纪九十年代中期闪亮登场。

    \item \textbf{深度学习} \\
    深度学习狭义地说就是"多层"神经网络,属于连接主义。虽然缺乏理论支持,但在模型复杂度高,参数调节好的情况下,深度学习地结果往往十分有效。\\
    深度学习地兴起与"大数据"和"计算能力提高"有密切关系。\\
\end{itemize}

\newpage